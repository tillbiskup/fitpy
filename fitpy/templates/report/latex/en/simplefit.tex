%{ extends templateDir + 'base.tex' }%

%{ block title }%
%{ endblock }%

%{ block body }%
\thispagestyle{empty}
\rule[1ex]{\textwidth}{1pt}
\\[1em]
\textbf{\LARGE Fit report}
\\[1ex]
\rule{\textwidth}{1pt}

\hfill --- \textsl{{@ sysinfo.user.login }}, {@ timestamp }

\section{Overview}

\begin{tabbing}
\hspace*{1.5cm}\=\kill
\textbf{Data}\\
Source:\>\texttt{{@ dataset.metadata.data.id | replace("_", "\\_") | replace("#", "\\#") }}
\\
Label:\>{@ dataset.metadata.data.label }
\\[1ex]
\textbf{Model}\\
Type:\>\texttt{{@ dataset.metadata.model.type }}
\\[1ex]
\textbf{Fit}\\
Type:\>\texttt{{@ dataset.metadata.calculation.type }}
\\
Success:\>%{ if dataset.metadata.result.success }% yes %{ else }% no %{ endif }%
\end{tabbing}

A simulation based on the model \texttt{{@ dataset.metadata.model.type }} has been
%{ if dataset.metadata.result.success }% successfully %{ else }% \textbf{unsuccessfully} %{ endif }%
fitted using FitPy's \texttt{{@ dataset.metadata.calculation.type }} class to the data of the dataset \enquote{{@ dataset.metadata.data.label }} (source: \texttt{{@ dataset.metadata.data.id | replace("_", "\\_") | replace("#", "\\#") }}). For a first graphical overview, cf. Fig.~\ref{fig:overview}. Further details of the fitting process and the underlying model are given below. Information on how this report has been generated and how to cite the underlying software are given at the end.


\begin{figure}[h!]
%{ if figureFilename }%
\begin{center}
\includegraphics{{@ figureFilename }}
\end{center}
%{ endif }%
\caption{\textbf{Overview of data and fitted simulation.} A simulation based on the model \texttt{{@ dataset.metadata.model.type }} has been
%{ if dataset.metadata.result.success }% successfully %{ else }% \textbf{unsuccessfully} %{ endif }%
fitted using FitPy's \texttt{{@ dataset.metadata.calculation.type }} class to the data of the dataset \enquote{{@ dataset.metadata.data.label }} (source: \texttt{{@ dataset.metadata.data.id | replace("_", "\\_") | replace("#", "\\#") }}). Further details of the fitting process and the underlying model are given below.}
\label{fig:overview}
\end{figure}

\clearpage

\section{Model}

\begin{tabbing}
\hspace*{1.5cm}\=\kill
Type:\>\texttt{{@ dataset.metadata.model.type }}
\end{tabbing}

%{ if dataset.metadata.model.parameters }%
The parameters of the model used for the final, fitted simulation as shown in Fig.~\ref{fig:overview}, are given in the table below.\footnote{Please note: Due to better compatibility with \LaTeX{}, the parameter names listed below have been changed from snake case (using the underscore \enquote{\_} as word separator) to camel case (medial capitals) with respect to their names in Python.} For further details regarding the fitting procedure, see the next section.

\vspace*{1ex}
\begin{tabular}{ll}
\toprule
\multicolumn{2}{c}{\textbf{Parameters}}
\\
\midrule
%{ for key, value in dataset.metadata.model.parameters.items() }%
{@ key | replace("_", "\\_") | replace("#", "\\#") } & {@ value | round(7) }
\\
%{ endfor }%
\bottomrule
\end{tabular}
\vspace*{1ex}
%{ else }%
For this model, no parameters were supplied nor necessary.
%{ endif }%


\section{Fitting}

\begin{tabbing}
\hspace*{1.5cm}\=\kill
Type:\>\texttt{{@ dataset.metadata.calculation.type }}
\\
Success:\>%{ if dataset.metadata.result.success }% yes %{ else }% no %{ endif }%
\\
Errors:\>%{ if dataset.metadata.result.error_bars }% yes %{ else }% no %{ endif }%
\end{tabbing}

\subsection{Fit results}

\begin{tabular}{lllllll}
\toprule
Parameter & start & min & max & end & stderr & varied
\\
\midrule
%{ for key, value in dataset.metadata.result.parameters.items() }%
{@ key } & {@ value.initValue } & {@ value.min } & {@ value.max } & {@ value.value | round(7) } & {@ value.stderr | round(7) } & %{ if value.vary }%yes%{else}%no%{ endif }%
\\
%{ endfor }%
\bottomrule
\end{tabular}


\subsection{Fit statistics}

\begin{tabbing}
\hspace*{5.5cm}\=\kill
Number of function evaluations:\> {@ dataset.metadata.result.nFunctionEvaluations }
\\
Degrees of freedom:\> {@ dataset.metadata.result.degreesOfFreedom }
\\
$\chi^2$\> {@ dataset.metadata.result.chiSquare | round(7) }
\\
reduced $\chi^2$\> {@ dataset.metadata.result.reducedChiSquare | round(7) }
\\
Akaike information criterion:\> {@ dataset.metadata.result.akaikeInformationCriterion | round(7) }
\\
Bayesian information criterion:\> {@ dataset.metadata.result.bayesianInformationCriterion | round(7) }
\end{tabbing}

Message of the solver:

\begin{quote}
{@ dataset.metadata.result.message }
\end{quote}

%{ include templateDir + "colophon.tex" }%
%{ endblock }%
